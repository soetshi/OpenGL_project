
This appendix specifies precisely what programmer visible state GLUT
maintains.  There are three categories of programmer visible state GLUT
maintains:  global, window, and menu.  The window and menu state categories
are maintained for each created window or menu.

The tables below name each element of state, define its type, specify
what GLUT API entry points set or change the state (if possible), specify what GLUT API
entry point or \rndx{glutGet} state constant is used to get the state (if possible),
and how the state is initially set.  For details of how any API entry point
operates on the specified state, see the routine's official description.
Notes for each category of state indicate additional caveats to the element
of state.

\subsection{Types of State}

The programmer visible state maintained by GLUT is of varying types:
\begin{description}

\item[\em BitMask]
A group of boolean bits.

\item[\em Boolean]
True or false.

\item[\em Callback]
A handle to a user-supplied routine invoked when the given callback is triggered
(or {\tt NULL} which is the default callback).

\item[\em ColorCell]
Red, green, and blue color component triple, an array of which makes
a colormap.

\item[\em Integer]
An integer value.

\item[\em MenuItem]
Either a menu entry or a submenu trigger.  Both subtypes contain
of a {\em String} name.  A menu entry has an {\em Integer} value.  A
submenu cascade has an {\em Integer} menu name naming its associated submenu.

\item[\em Stacking]
An ordering for top-level windows and sub-windows having the same
parent.  Higher windows obscure lower windows.

\item[\em State]
One of shown, hidden, or iconified.

\item[\em String]
A string of ASCII characters.

\item[\em Timer]
A triple of a timer {\em Callback}, an {\em Integer} callback parameter, and a time
in milliseconds (that expires in real time).

\end{description}

%\newsavebox{\globalstate}

\subsection{Global State}

{\tiny
\begin{tabular}{|l|l|l|l|l|} \hline
Name & Type & Set/Change & Get & Initial \\ \hline
currentWindow & Integer & glutSetWindow \takeNote & glutGetWindow & 0 \\
currentMenu & Integer & glutSetMenu \takeNote & glutGetMenu & 0 \\
initWindowX & Integer & glutInitWindowPosition & GLUT\_INIT\_WINDOW\_X & -1 \\
initWindowY & Integer & glutInitWindowPosition & GLUT\_INIT\_WINDOW\_Y & -1 \\
initWindowWidth & Integer & glutInitWindowSize & GLUT\_INIT\_WINDOW\_WIDTH & 300 \\
initWindowHeight & Integer & glutInitWindowSize & GLUT\_INIT\_WINDOW\_HEIGHT & 300 \\
initDisplayMode & BitMask & glutInitDisplayMode & GLUT\_INIT\_DISPLAY\_MODE & GLUT\_RGB, \\
& & & & GLUT\_SINGLE, \\
& & & & GLUT\_DEPTH \\
idleCallback & Callback & glutIdleFunc & - & NULL \\
menuStateCallback & Callback & glutMenuEntryFunc & - & NULL \\
timerList & list of Timer & glutTimerFunc & - & none \\
screenWidth & Integer & - & GLUT\_SCREEN\_WIDTH & system dependent \\
screenHeight & Integer & - & GLUT\_SCREEN\_HEIGHT & system dependent \\
screenWidthMM & Integer & - & GLUT\_SCREEN\_WIDTH\_MM & system dependent \\
screenHeightMM & Integer & - & GLUT\_SCREEN\_HEIGHT\_MM & system dependent \\
\hline
\end{tabular}
}

\resetNote

{\footnotesize
\begin{description}
\itemsep 0in
\item[\takeNote] The currentWindow is also changed implicitly by every window
callback (to the window triggering the callback) and the creation of a window
(to the window being created).
\item[\takeNote] The currentMenu is also changed implicitly by every menu
callback (to the menu triggering the callback) and the creation of a menu
(to the menu being created).
\end{description}
}

%\input rotate.tex
%\def\myrotr#1{\rotdimen=\ht#1\advance\rotdimen by\dp#1%
%   \hbox to\rotdimen{\vbox to\wd#1{\vskip\wd#1%
%   \rotstart{270 rotate}%
%\box#1\vss}\hss}\rotfinish}%
%
%\newbox\thebox
%
%\setbox\thebox=\vbox{\usebox{\globalstate}}
%
%\myrotr\thebox

\subsection{Window State}

For the purposes of listing the window state elements, window state is
classified into two types:  generic state and frame buffer capability state.
The tags {\em top-level}, {\em sub-win}, and {\em cindex} indicate the
table entry applies only to top-level windows, subwindows, or color index
windows respectively.

\subsubsection{Generic State:}

\resetNote
{\tiny
\begin{tabular}{|l|l|l|l|l|} \hline
Name & Type & Set/Change & Get & Initial \\ \hline
number & Integer & - & glutGetWindow & {\em top-level:} glutCreateWindow \takeNote \\
& & & & {\em sub-win:} glutCreateSubWindow \sameNote \\
x & Integer & glutPositionWindow & GLUT\_WINDOW\_X & {\em top-level:} initWindowX \takeNote \\
& & & & {\em sub-win:} glutCreateSubWindow \\
y & Integer & glutPositionWindow & GLUT\_WINDOW\_Y & {\em top-level:} initWindowY \takeNote \\
& & & & {\em sub-win:} glutCreateSubWindow \\
width & Integer & glutReshapeWindow & GLUT\_WINDOW\_WIDTH & {\em top-level:} initWindowWidth \takeNote \\
& & & & {\em sub-win:} glutCreateSubWindow \\
height & Integer & glutReshapeWindow & GLUT\_WINDOW\_HEIGHT & {\em top-level:} initWindowHeight \takeNote \\
& & & & {\em sub-win:} glutCreateSubWindow \\
toplevel: fullScreen & Boolean & glutFullScreen & & False \\
& & glutPositionWindow & & \\
& & glutReshapeWindow \takeNote & & \\
stacking & Stacking & glutPopWindow & - & top \\
& & glutPushWindow & & \\
displayState & State \takeNote & glutShowWindow \takeNote & - & shown \\
& & glutHideWindow & & \\
& & glutIconifyWindow & & \\
visibility & Visibility & \takeNote & \takeNote & undefined \\
redisplay & Boolean & glutPostRedisplay \takeNote & - & False \\
toplevel: windowTitle & String & glutWindowTitle & - & glutCreateWindow \\
toplevel: iconTitle & String & glutIconTitle & - & glutCreateWindow \\
displayCallback & Callback & glutDisplayFunc & - & NULL \\
reshapeCallback & Callback & glutReshapeFunc & - & NULL \takeNote \\
keyboardCallback & Callback & glutKeyboardFunc & - & NULL \\
mouseCallback & Callback & glutMouseFunc & - & NULL \\
motionCallback & Callback & glutMotionFunc & - & NULL \\
passiveMotionCallback & Callback & glutPassiveMotionFunc & - & NULL \\
specialCallback & Callback & glutSpecialFunc & - & NULL \\
spaceballMotionCallback & Callback & glutSpaceballMotionFunc & - & NULL \\
spaceballRotateCallback & Callback & glutSpaceballRotateFunc & - & NULL \\
spaceballButtonCallback & Callback & glutSpaceballButtonFunc & - & NULL \\
buttonBoxCallback & Callback & glutButtonBoxFunc & - & NULL \\
dialsCallback & Callback & glutDialsFunc & - & NULL \\
tabletMotionCallback & Callback & glutTabletMotionFunc & - & NULL \\
tabletButtonCallback & Callback & glutTabletButtonFunc & - & NULL \\
visibilityCallback & Callback & glutVisibilityFunc & - & NULL \\
entryCallback & Callback & glutEntryFunc & - & NULL \\
{\em cindex:} colormap & array of & glutSetColor & glutGetColor & undefined \\
& ColorCell & & & \\
windowParent & Integer & - & GLUT\_WINDOW\_PARENT & {\em top-level:} 0 \\
& & & & {\em sub-win:} \takeNote \\
numChildren & Integer & glutCreateSubWindow & GLUT\_NUM\_CHILDREN & 0 \\
& & glutDestroyWindow & & \\
& & glutCopyColormap & & \\
leftMenu & Integer & glutAttachMenu & - & 0 \\
& & glutDetachMenu & & \\
middleMenu & Integer & glutAttachMenu & - & 0 \\
& & glutDetachMenu & & \\
rightMenu & Integer & glutAttachMenu & - & 0 \\
& & glutDetachMenu & & \\
\hline
\end{tabular}
}

\resetNote

{\footnotesize
\begin{description}
\itemsep 0in
\item[\takeNote]  Assigned dynamically from unassigned window numbers greater than zero.
\item[\takeNote]  If initWindowX is greater or equal to zero {\em and}
initWindowY is greater or equal to zero then initWindowX, else
window location left to window system to decide.
\item[\takeNote]  If initWindowY is greater or equal to zero {\em and}
initWindowX is greater or equal to zero then initWindowY, else
window location left to window system to decide.
\item[\takeNote]  If initWindowWidth is greater than zero {\em and} initWindowHeight is greater than zero
the initWindowWidth, else
window size left to window system to decide.
\item[\takeNote]  If initWindowHeight is greater than zero {\em and} initWindowWidth is greater than zero
then initWindowHeight, else
window size left to window system to decide.
\item[\takeNote]  {\tt glutFullScreen} sets to true; {\tt glutPositionWindow} and
{\tt glutReshapeWindow} set to false.
\item[\takeNote]  Subwindows can not be iconified.
\item[\takeNote]  Window system events can also change the displayState.
\item[\takeNote]  Visibility of a window can change for window system dependent reason, for
  example, a new window may occlude the window.  glutPopWindow and glutPushWindow can
  affect window visibility as a side effect.
\item[\takeNote]  The visibility callback set by glutVisibilityFunc allows the visibility state to be tracked.
\item[\takeNote]  The redisplay state can be explicitly enabled by glutRedisplayFunc or
  implicitly in response to redisplay events from the window system.
\item[\takeNote]  Instead of being a no-op as most NULL callbacks are, a NULL reshapeCallback
sets the OpenGL viewport to render into the complete window, ie. {\tt glViewport(0,0,width, height)}.
\item[\takeNote]  Determined by currentWindow at glutCreateSubWindow time.
\end{description}
}

\subsubsection{Frame Buffer Capability State:}

\resetNote
{\tiny
\begin{tabular}{|l|l|l|} \hline
Name & Type & Get \\ \hline
Total number of bits in color buffer & Integer & GLUT\_WINDOW\_BUFFER\_SIZE \\
Number of bits in stencil buffer & Integer & GLUT\_WINDOW\_STENCIL\_SIZE \\
Number of bits in depth buffer & Integer & GLUT\_WINDOW\_DEPTH\_SIZE \\
Number of bits of red stored in color buffer & Integer & GLUT\_WINDOW\_RED\_SIZE \\
Number of bits of green stored in color buffer & Integer & GLUT\_WINDOW\_GREEN\_SIZE \\
Number of bits of blue stored in color buffer & Integer & GLUT\_WINDOW\_BLUE\_SIZE \\
Number of bits of alpha stored in color buffer & Integer & GLUT\_WINDOW\_ALPHA\_SIZE \\
Number of bits of red stored in accumulation buffer & Integer & GLUT\_WINDOW\_ACCUM\_RED\_SIZE \\
Number of bits of green stored in accumulation buffer & Integer & GLUT\_WINDOW\_ACCUM\_GREEN\_SIZE \\
Number of bits of blue stored in accumulation buffer & Integer & GLUT\_WINDOW\_ACCUM\_BLUE\_SIZE \\
Number of bits of alpha stored in accumulation buffer & Integer & GLUT\_WINDOW\_ACCUM\_ALPHA\_SIZE \\
Color index colormap size & Integer & GLUT\_WINDOW\_COLORMAP\_SIZE \\
If double buffered & Boolean & GLUT\_WINDOW\_DOUBLEBUFFER \\
If RGBA color model & Boolean & GLUT\_WINDOW\_RGBA \\
If stereo & Boolean & GLUT\_WINDOW\_STEREO \\
Number of samples for multisampling & Integer & GLUT\_WINDOW\_MULTISAMPLE \\
\hline
\end{tabular}
}

A window's frame buffer capability state is derived from the global initDisplayMode
state at the window's creation.  A window's frame buffer capabilities can not
be changed.

\subsection{Menu State}

\resetNote

{\tiny
\begin{tabular}{|l|l|l|l|l|} \hline
Name & Type & Set/Change & Get & Initial \\ \hline
number & Integer & - & glutSetMenu & {\em top-level:} glutCreateMenu \takeNote \\
select & Callback & - & - & glutCreateMenu \\
items & list of MenuItem & - & - & - \\
numItems & Integer & - & GLUT\_MENU\_NUM\_ITEMS & 0 \\
\hline
\end{tabular}
}
\resetNote

{\footnotesize
\begin{description}
\itemsep 0in
\item[\takeNote]  Assigned dynamically from unassigned window numbers greater than zero.
\end{description}
}

